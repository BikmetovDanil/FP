\documentclass[a4paper, 12pt, oneside]{scrartcl}
%\documentclass[a4paper, 12pt, oneside]{ncc}
\usepackage[warn]{mathtext}          % русские буквы в формулах, с предупреждением
\usepackage[T2A]{fontenc}            % внутренняя кодировка  TeX
\usepackage[utf8x]{inputenc}         % кодовая страница документа
\usepackage[english, russian]{babel} % локализация и переносы
\usepackage{indentfirst}   % русский стиль: отступ первого абзаца раздела
\usepackage{misccorr}      % точка в номерах заголовков
\usepackage{cmap}          % русский поиск в pdf
\usepackage{graphicx}      % Работа с графикой \includegraphics{}
\usepackage{psfrag}        % Замена тагов на eps картинкаx
\usepackage{caption2}      % Работа с подписями для фигур, таблиц и пр.
\usepackage{soul}          % Разряженный текст \so{} и подчеркивание \ul{}
\usepackage{soulutf8}      % Поддержка UTF8 в soul
\usepackage{fancyhdr}      % Для работы с колонтитулами
\usepackage{multirow}      % Аналог multicolumn для строк
\usepackage{amsmath, amssymb}

\begin{document}

\section{Алгоритм AD}

Алгоритм AD реализован на языке C++. На вход подается бинарная матрица и минимальная поддержка. В текущей реализации используется не чисто бинарная матрица (как у П. А. Прокофьева в RUNC-M), а матрица из байтов. Пример работы алгоритма:

\subsection{Пример 1}

\textbf{Вход:}
\begin{equation}
\begin{pmatrix} 
  0 & 1 & 0 & 1 & 1 & 1\\ 
  0 & 1 & 0 & 1 & 1 & 0\\ 
  0 & 1 & 1 & 1 & 1 & 0\\ 
  1 & 0 & 1 & 1 & 1 & 0\\ 
  0 & 0 & 1 & 1 & 1 & 1\\ 
  1 & 1 & 1 & 1 & 0 & 0
\end{pmatrix}
\end{equation}

\quad Минимальная поддержка s = 3.

\textbf{Выход:}

\quad Всего наборов: 11

\qquad 1

\qquad 1 3

\qquad 1 3 4

\qquad 1 4

\qquad 2 3

\qquad 2 3 4

\qquad 2 4

\qquad 3

\qquad 3 4

\qquad 4

\quad Максимальных наборов: 2

\qquad 1 3 4

\qquad 2 3 4

\subsection{Пример 2}

\textbf{Вход:}
\begin{equation}
\begin{pmatrix} 
  0 & 1 & 1 & 1 & 0 & 0\\ 
  1 & 0 & 0 & 1 & 1 & 1\\ 
  0 & 0 & 0 & 1 & 1 & 1\\ 
  1 & 1 & 0 & 1 & 1 & 1\\ 
  1 & 0 & 0 & 0 & 0 & 1\\ 
  0 & 0 & 0 & 0 & 1 & 0
\end{pmatrix}
\end{equation}

\quad Минимальная поддержка s = 3.

\textbf{Выход:}

\quad Всего наборов: 9

\qquad 0

\qquad 0 5

\qquad 3

\qquad 3 4

\qquad 3 4 5

\qquad 3 5

\qquad 4

\qquad 4 5

\qquad 5

\quad Максимальных наборов: 2

\qquad 0 5

\qquad 3 4 5

\section{Скорость счёта}

Множество всех s-совместимых наборов строилось строго по алгоритму. Множество максимальных наборов строилось тогда, когда алгоритм не мог найти следующий совместимый столбец. В этом случае максимальность проверялась по базе данных (модификация была предложена Н. А. Драгуновым в курсовой работе). Результаты счёта приведены в таблице:

\begin{table}[h!]
\begin{tabular}{|c|c|c|}
\hline
                  & \textbf{n = 20}                                        & \textbf{n = 30}                                         \\ \hline
\textbf{m = 100}  & \begin{tabular}[c]{@{}c@{}}747,\\ 72 мс\end{tabular}   & \begin{tabular}[c]{@{}c@{}}3400,\\ 489 мс\end{tabular}  \\ \hline
\textbf{m = 1000} & \begin{tabular}[c]{@{}c@{}}1136,\\ 764 мс\end{tabular} & \begin{tabular}[c]{@{}c@{}}4042,\\ 3649 мс\end{tabular} \\ \hline
\end{tabular}
\end{table}

Здесь m - число транзакций (строк), n - число атрибутов (столбцов). Минимальная поддержка s = 0.1. В первой строке указано число всех максимальных частых наборов, во второй строке - время работы алгоритма. Сравнивая с результатами Н. А. Драгунова можно сделать вывод, что алгоритм AD работает быстрее DepthProject без модификации, но медленнее DepthProject с модификацией. Однако текущая реализация алгоритма AD далека от оптимальной, и есть надежда, что результаты можно улучшить.

\end{document}